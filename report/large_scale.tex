\section{Large scale image retrieval}

Nous nous sommes également intéressés à certaines transformations qui aident largement à déployer un système d'image
retrieval à grande échelle. Dans ce cas, la précision n'est plus le seul critère: il faut également prendre en compte le
temps de réponse, la vitesse d'exécution ainsi que respecter un espace de stockage restreint. Nous nous sommes donc
penchés sur la binarisation de features, ce qui permet de réduire grandement la taille des features que nous utilisons.
Enfin, nous allons étudier des outils de compression de features afin de voir quelles sont les pertes de performances,
et à quel coût.

\section{Pistes d'études}

Au vu de nos bonnes performances sur le dataset CIFAR-10 nous avons cherché à appliquer nos méthodes à de nouveaux
datasets. Un des datasets qui semble intéressant est celui des bâtiments
d’Oxford \cite{object2007philbin}. Il comporte 10 classes et environ 2000 images avec des
labels et de bonnes qualités. Malheureusement nous n’avons pas réussi à avoir des résultats satisfaisants. Nous avons en
fait réussi à faire converger le réseau siamois mais nous n’avons pas réussi à faire marcher l’entrainement du réseau de
triplets. Nous plafonnons donc à une précision à 10 de 0.15 que ce soit avec la représentation latente d’image net ou
avec celle du réseau siamois. Nous avons essayé de changer de réseau et d’essayer avec ResNet50 sans succès. La revue de
la littérature autour de ce problème semble indiquer qu’il s’agit plutôt d’un dataset de test que d’entrainement du fait
de la difficulté du problème et du petit nombre d’image.

D’après \cite{end2017gordo}, il semblerait qu’il faille pré-entrainer le réseau de triplet sur un autre dataset de
bâtiments comme celui-ci utilisé dans \cite{neural2014babenko}  ou utiliser un réseau plus profond comme ResNet101
\cite{deep2015he}. Malheureusement, les poids du réseau pré-entrainé ne sont pas disponible dans Keras.

En plus, d’essayer nos modèles sur d’autres datasets il aurait été intéressant d’explorer d’autres techniques éprouvées
sur ce problème particulier comme l’architecture R-Mac \cite{particular2015tolias}.


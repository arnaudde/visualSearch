\section*{Introduction}

Les algorithmes de recommandations sont au coeur de la stratégie des entreprises de e-commerce afin de proposer à leurs
clients des produits qui pourraient correspondre à leurs besoins. Ces algorithmes de recommandations, déjà éprouvés, se
basent souvent sur l'établissement d'un profil utilisateur en fonction de son comportement sur le site : clics, pages
visitées et produits achetés. Ce comportement est ensuite comparé à la liste des produits et aux comportements des
autres clients à travers différentes méthodes, comme le "collaborative filtering" ou des modèles de clusters
\cite{amazon2003linden}. Cependant, ces méthodes souffrent de nombreux défauts, comme un coût de calcul prohibitif
empêchant un passage à grande échelle, ainsi qu'une précision faible \cite{deep2017shankar}.

Enfin, un autre problème majeur de ces techniques est qu'elles reposent souvent sur des métadonnées ou des données
textuelles. Or, dans le domaine de la mode, les utilisateurs souhaitent pouvoir chercher des produits et obtenir des
recommandations basées sur des critères visuels. Il est donc crucial pour les sites de e-commerce, qui possèdent de
grandes bases de données d'images mais non étiquetées, de pouvoir proposer des systèmes de recommandations basés sur
ces critères visuels afin d’atteindre un “shazam des produits”. Ce projet a pour objectif de faire les premiers pas
dans cette direction. 


\section*{Introduction}

De nombreux sites de e-commerce ont à leur disposition des bases de données d’images. Cependant celles-ci sont souvent
non-étiquetées et il est intéressant de pouvoir identifier les images de produits qui sont similaires. Le graal de cette
recherche serait d’atteindre un “shazam des produits”. Ce projet a pour objectif de faire les premiers pas dans cette
direction. 

Grâce aux récents développements des techniques de Machine Learning et de Deep Learning, les systèmes de classification
et de détection d’images sont maintenant à des niveaux de performance égalant les humains sur certaines tâches. Ils
permettent d’extraire de manière automatique les informations essentielles caractérisant les images.

Le coeur des techniques de matching repose sur l’apprentissage d’une représentation latente des images. C’est ainsi dans
cet espace que l’étude de similarité est alors possible.

Nous allons tout d'abord travailler sur l'apprentissage d'une fonction de similarité entre images, afin de nous attaquer
au cas d'image matching.

\subsection{État de l'art}



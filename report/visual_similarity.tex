\section{Visual Similarity}

Dans le but de faire de l'"Image Retrieval", nous somme d'abord pench\'e sur le probl\`eme d'apprendre une similarit\'e
entre images. Nous avons pour cela \'evalu\'e diff\'erentes m\'ethodes, que nous allons pr\'esenter ci-dessous.

\subsection{Extraction de features}

\subsection{R\'eseaux Siamois}
 
\begin{figure}[ht]
 \center
 \includegraphics[width = 0.8\textwidth]{figures/siamnet_arch}
 \caption{\label{siamnet_arch} \textbf{Architecture d'un r\'eseau siamois} Les poids des deux r\'eseaux sont li\'es. L'avant derni\`ere
    couche induit une m\'etrique apprise sur l'espace des features apprises. \textit{p} est une pr\'ediction logistique.
    \cite{koch2015siamese}}
\end{figure}

\subsection{R\'eseaux Triplets}

Nous avons aussi essayé d'utiliser un \textit{réseau triplet}, qui ont été introduits pour la première fois par
\cite{hoffer2014deep}.

\begin{figure}[ht]
    \center
    \includegraphics[width = 0.6\textwidth]{figures/triplet_net}
    \caption{\label{siamnet_arch} \textbf{Architecture d'un réseau triplet - changer d'image}
    \cite{hoffer2014deep}}
\end{figure}

Les réseaux triplets sont une amélioration des réseaux siamois déjà susmentionnés. À chaque itération, le nombre
d'images en entrée est de 3 : 

\begin{itemize}
    \item image de base
    \item image similaire
    \item image différente
\end{itemize}

Cela revient à ajouter un nouveau terme dans la fonction de perte du modèle. Cet ajout permet 
